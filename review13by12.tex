\documentclass[10pt,a4paper]{report} 
% Style
\usepackage{amsfonts}
\usepackage{amsmath}
\usepackage{amssymb}
\usepackage[utf8]{inputenc}
\usepackage[T1]{fontenc}
\usepackage[english]{babel}
\usepackage{lmodern} 
\usepackage{graphicx}
\usepackage{color}
\usepackage{float}
\usepackage{url}
\usepackage[numbered]{mcode}
\usepackage[top=4cm, bottom=4cm, left=4cm, right=4cm]{geometry}

\lstset{extendedchars=\true}
\lstset{inputencoding=ansinew}



% ---Dokumen start---
\begin{document}



% ---Title page ---
\title{Peer review on group 13 by group 12\\Sensor fusion -- TSRT14}
\author{Group 12}
\maketitle

\newpage
\section*{1}
The datasets are well described, but the exact positioning of the sensors for the ''good'' and the ''bad'' alternative should be given explicitly here.
Also a figure of the setup would be great.
\begin{verbatim*}
plot(sensormodel)
\end{verbatim*}
The procedures to acquire the the data are well described as well, 
but the used commands in matlab during the collection should also be mentioned, 
otherwise it will be hard for someone who has never done this, to actually acquire the data without fault.



\section*{2}
\subsection*{Task 1: Sensor calibration}
Explained in the text and visualized with figure 1.1. 
Since it was an occurring issue, it would be good to clearly define the used values as distance or time. 
The values of the standard deviation and bias are clearly presented, although the standard deviation is called ''$varvalues\_e$'', 
which might be confusing, since the standard deviation is $\sqrt{(var(e))}$ and not $var(e)$.
There is no unit used and the mean values seems to be very small.

\subsection*{Task 2: Signal modeling}
Sufficient description of used equations, matlab functions and model approaches. 
Also the colouring scheme of the matlab code is kept, which is a nice touch.

\subsection*{Task 3: Experiments}
All element in the figures are clearly visible, the axis have labels and the figures have good captions. 
For anyone who is unfamiliar with the toolbox, it would be helpful to have the code used to produce these figures.
There is no description of the sendor model variances used.

\subsection*{Task 4: Configuration analysis}
The solutions are well described, but a excerpt of the code used to produce the figures would be helpful to anyone not familiar with the toolbox. 
The figures are very good, clearly visible graphs and values.
No explanation of how the figures should be interpreted, except the conclusion of which one is the best.

\subsection*{Task 5: Localisation}
Also very intuitive described solutions, but lacking code excerpts to illustrate how the task was solved. 
Optionally use 'SFlabCompEstimGroundTruth' to plot the target estimates, but the provided figures are also visualising the targets position very well.

\subsection*{Task 6: Tracking}
Again, good description, but there should really be pieces of code showing how to solve this problem.
The motion models used are named but there is no mathematical description of them.
There is also no mention of the actual paramaters used after the tuning of the model.
There is no mention of the variance from the estimated positions, how are these used in the tracking assignment.

\subsection*{Task 7: Sensitivity analysis}
The task solution is well explained and the figure show a clear result of the disturbed sensormodel compared to the one from the previous exercise. 
To make the comparison even better, calculate the standard deviation between the two models, to help putting a number to the sensitivity.

\section*{3}
Generally the conclusions were supported by data and plots or results.
There is though little support in the text and the reader is expected to know how to use the plots.
The conclusion are otherwise good and seems accurate.

\section*{4}
The strength of this report is mainly that it is short, but still contains most of what neccessary to know what was done.
The plots are captioned and reffered to in the text.
For are reader that has done similar work the plots are enough to support the conclusions of the report.

\section*{5}
To make the report better there are a few things that can be improved.
There are a lot of plots and they are captioned, but there is very few explanation of how they should be interpreted.
Some of them have numbers in them that aren't explained (figure 8,9 and 10).

The histogram in figure 1 is mentioned in the text but it is not certain what it actually shows.
There is one histogram and many plotted distributions, how are they related to each other.
It is stated that the shown histogram 'differs a bit' from a gaussian distribution.
The plot show something that differs a lot from a gaussian distribution.
The estimated means are also strange, they seem very very small.
There is also no unit mentioned for them.

The conclusions are clear and it is for example stated which model or sensor setup was the best, but the conclusions are not very well supported in the text.
The reason why a figure supported a certain conclusion could be mentioned, not just that it does.

There could also be improvements in the explanation of how the tracking assignment.
The motion models and the parameters used could be included for more technical accuracy.

Generally it is assumed that the reader of the report has a lot of prior knowledge as many of the tools are used with no explanation of them.
The readability is fine, the text is easy to read and the conclusion are clear.
What is lacking is mainly the technical accuracy.
How should plots be interpreted.
How is the different variances used and estimated?
What are for example SLS, WLS and the loss function?

\end{document}



%%% Local Variables: 
%%% TeX-PDF-mode: t 
%%% TeX-master: "rapport"
%%% End: 

